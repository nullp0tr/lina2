\documentclass{article}
\title{what eves}
\usepackage[utf8]{inputenc}
\usepackage[english]{babel}
\usepackage{amsmath}
\usepackage{amsthm}
\usepackage{amssymb}
\usepackage{hyperref}
\usepackage{tcolorbox}

\newtheorem*{theorem}{Satz 22.7}

\newtheorem*{example}{Beispiel}

\renewcommand \qedsymbol{$\square$}

\newcommand{\LL}{\mathcal{L}}
\newcommand{\V}{\mathcal{V}}
\newcommand{\U}{\mathcal{U}}
\newcommand{\N}{\mathbb{N}}
\newcommand{\K}{\mathbb{K}}
\newcommand{\M}{\mathbb{M}}

\newcommand{\basis}[2]{b_{#1}^{({#2})}}
\newcommand{\basislist}[2]{\basis{{#1}}{1}, \dots , \basis{{#1}}{{#2}}}
\newcommand{\basislistbot}[2]{\basis{1}{{#2}}, \dots , \basis{{#1}}{{#2}}}

\newcommand{\lam}{\lambda}

\newcommand{\flam}[1]{(f-{\lambda}id)^{{#1}}}

\newcommand{\funcdown}[2]{{{#1}}_{|{#2}}\colon{#2}\to{#2}}

\newcommand{\betw}[3]{{#1} \le {#2} \le {#3}}

\newcommand{\jmatrix}{
  \begin{bmatrix}
    \lam   & 1     & 0      & \dots & 0      \\
    0      & \lam  & 1      & \dots & 0      \\
    \vdots &       & \ddots &       & \vdots \\
    0      & \dots & \dots  & \lam  & 1      \\
    0      & \dots & \dots  & \dots & \lam
  \end{bmatrix}}


\DeclareMathOperator{\Kern}{Kern}
\DeclareMathOperator{\Bild}{Bild}
\DeclareMathOperator{\dimension}{dim}
\DeclareMathOperator{\lin}{lin}

\begin{document}

\begin{theorem}
  Seien $\V$ ein Vektorraum, $\dim \V = n$, $ f \in \LL (\V;\V)$ und $\lambda$ eine n-fache
  Nullstelle von der char. Polynom $P(t)$. Dann gibt es $r,k_j,q_j \in \N,
  1 \le j \le r$ mit $\sum_{j=1}^{r}k_jq_j=n$, und eine Matrixdarstellung
  $J(\lambda)$ von f der Form

  \begin{equation*}
    J(\lam) = 
    \begin{bmatrix}
      J_{q_1}(\lambda) &         &                  &                  &        &  0 \\
                      & \ddots  &                  &                  &        & \\
                      &         & J_{q_1}(\lambda)  &                  &        & \\
                      &         &                  & J_{q_2}(\lambda)  &        & \\
                      &         &                  &                  & \ddots & \\
      0               &         &                  &                  &        & J_{q_r}(\lambda)
    \end{bmatrix},
  \end{equation*}
  wobei die Jordanmatrizen $J_{q_j}(\lam)$ $k_j$-mal wiederholt werden.
  
\end{theorem}

\begin{proof}[Beweis]
  Nach Cayley-Hamilton gilt $\flam{n}=0 \in \LL(\V;\V)$, und somit $V_f(\lam)=V$.

  Sei nun $q_1 \in \{1, .., n\}$ minimal mit der Eigenschaft $(f -
  \lambda id)^{q_1}=0$, d.h. $(f-{\lambda}id)^{q_1-1}\ne0$.
  Seien weiter $\U:=(f-{\lambda}id)^{q_1-1}(V)$ und $\dimension(\U) =: k_1$.

  Wähle $b_1, .., b_{k_1}\in V$ sodass die Menge
  $(f-{\lambda}id)^{q_1-1}(\{b_1, ..,b_{k_1}\})$ eine Basis von $\U$
  bildet.

  Per Konstruktion ist jedes $u \in \U \backslash \{0\}$ ein Eigenvektor zu
  $\lambda$, denn ist $u=\sum_{i=0}^{k_1}(f-{\lambda}id)^{q_1-1}(b_i)$
  so gilt:

  \begin{align*}
    \displaystyle
    f(u) & = f(\sum_{i=0}^{k_1}(f-{\lambda}id)^{q_1-1}(b_i)) 
          = \sum_{i=0}^{k_1}f((f-{\lambda}id)^{q_1-1}(b_i)) \\
         & = \sum_{i=0}^{k_1}(f-{\lambda}id)^{q_1-1}(f(b_i)) 
          = \sum_{i=0}^{k_1}((f-{\lambda}id)^{q_1-1} \circ
           (f-{\lambda}id+{\lambda}id))(b_i) \\
         & = \sum_{i=0}^{k_1}(f-{\lambda}id)^{q_1}(b_i) +
           (f-{\lambda}id)^{q_1-1}({\lambda}b_i)
           = \lambda \sum_{i=0}^{k_1}(f-{\lambda}id)^{q_1-1}(b_i) \\
         & = \lambda \cdot u.
  \end{align*}

  Für $0 \le i \le q_1, 1 \le j \le k_1$ definiere
  $b_j^{(i)}:=(f-{\lambda}id)^{q_1-i}(b_i)$. Dann ist $(b_1^{(1)}, ..,
  b_{k_1}^{(1)})$ eine Basis von $\U$.

  Für jedes $j \in \{1, \dots ,k_1\}$ sind die $q_1$ Vektoren
  $\basis{j}{1} , \dots , \basis{j}{q_1}$ nach Satz 22.4 linear
  unabhängig, und es gilt:

  \begin{align*}
    f(\basis{j}{i}) & = f(\flam{q_1-i}(b_j)) = \flam{q_1-i}(f(b_j)) \\
                    & = \flam{q_1-i}(f(b_j) - \lam b_j + \lam b_j) \\
                    & = (\flam{q_1-i} \circ (f - \lam id + \lam id))(b_j) \\
                    & = \flam{q_1-i+1}(b_j) + (f-\lam id)(\lam b_j) \\
                    & = \basis{j}{i-1} + \lam \basis{j}{i}.
  \end{align*}

  Mit $\U_j = \lin \{\basis{j}{1}, \dots , \basis{j}{q_1}\}, 1 \le j \le k_1$ folgt, dass
  die Abbildung $\funcdown{f}{\U_j}$ bzgl. der Basis $\basis{j}{1}, \dots ,\basis{j}{q_1}$
  die Matrixdarstellung $J_{q_1}(\lam)$ hat. 
  
  \begin{tcolorbox}
    Erinnerung: Für $k \ge 2$ lässt sich die k-te Spalte ${A}_k$ der Matrixdarstellung ${A}$ von $\funcdown{f}{\U_j}$ wie Folgendes berechnen:

    \begin{align*}
      A_k & = (B \circ f \circ B^{-1})(e_k) 
            = (B \circ f)(\basis{j}{k}) \\
          & = B(\basis{j}{k-1} + \lam \basis{j}{k})
           = B(\basis{j}{k-1}) + \lam B(\basis{j}{k}) \\
          & = e_{k-1} + \lam e_k
    \end{align*}

    wobei $B \colon \U_j \to \U_j$ der lineare Isomormphismus ist, der die Basisvektoren
    $\basislist{j}{q_1}$ auf die kanonischen Basisvektoren $e_1, \dots , e_{q_1}$ abbildet.
    Für $k=1$ gilt $B(\basis{j}{0}) = B(0) = 0$ und somit $A_1 = \lam e_1$. Folglich gilt:
    \begin{equation*}
      A = \jmatrix = J_{q_1}(\lam).
    \end{equation*}

  \end{tcolorbox}

  Weiter gilt, dass $\basis{j}{i}, 1 \le j \le k_1$ für alle $1 \le i \le q_1$ linear
  unabhängig sind, denn sind $\mu_{j}^{(i)} \in \K$ mit
  $0 = \sum_{i}^{q_1}\sum_{j}^{k_1}\mu_j^{(i)}\basis{j}{i} = \sum_{i,j}\mu_j^{(i)}\basis{j}{i}$
  so folgt:

  \begin{align*}
    0 & = \flam{q_1-1}(\sum_{i,j}\mu_j^{(i)}\basis{j}{i})
        = \sum_{i,j}\mu_j^{(i)} \flam{q_1-1}(\basis{j}{i}) \\
      & = \sum_{i,j}\mu_j^{(i)} \flam{q_1-1}(\flam{q_1-i}(b_j))
        = \sum_{i,j}\mu_j^{(i)} \flam{q_1-1+q_1-i}(b_j) \\
      & = \sum_{j=1}^{k_1}\mu_j^{(q_1)} \flam{q_1-1}(b_j)
        = \sum_{j=1}^{k_1}\mu_j^{(q_1)} \basis{j}{1}.
  \end{align*}

  Da $\basislistbot{k_1}{1}$ nach Wahl eine Basis von $\U$ bilden, folgt,
  dass $\mu_j^{(q_1)} = 0$. Analog zeigt man, dass
  $\mu_{j}^{(q_1-1)} = \mu_{j}^{(q_1-2)} = \dots = \mu_{j}^{(1)} = 0$.

  Somit ist $V_1 := \U_1 + \U_2 + \dots + \U_{k_1}$ ein Vektorraum mit $\dim V_1 = q_1k_1$.
  Daraus folgt, dass $\funcdown{f}{V_1}$ bzgl. der Basis
  $(\basis{j}{i}), 1 \le i \le q_1, 1 \le j \le k_1$ eine Matrixdarstellung $A$ hat, die aus
  $k_1$ Jordanmatrizen der From $J_{q_1}(\lam)$ besteht.

  \begin{tcolorbox}
    Also gilt:
    \begin{align*}
      A & =
          \begin{bmatrix}
            J_{q_1}(\lam) & & 0 \\
            & \ddots & \\
            0 & & J_{q_1}(\lam) \\
          \end{bmatrix} \\
        & =
          \begin{bmatrix}
            \jmatrix & & 0 \\
            & \ddots & \\
            0 & & \jmatrix
          \end{bmatrix}
    \end{align*}

    Beachte dass, $a_{nq_1,nq_1+1}=0$ für alle $1 \le n \le k_1-1$, denn es gilt:
    \begin{equation*}
      f(\basis{1}{1}) = 0 + {\lam} \basis{1}{1} ,
      f(\basis{2}{1}) = 0 + {\lam} \basis{2}{1} ,
      \dots ,
      f(\basis{k_1}{1}) = 0 + {\lam} \basis{k_1}{1}.
    \end{equation*}

  \end{tcolorbox}

  Ist $V_1 = \V$ so folgt die Aussage des Satzes sofort. Sei also $V_1 \ne \V$. Daraus folgt $q_1 \ge 2$.

  \begin{tcolorbox}
    Ist $q_1 = 1$, so folgt:
    \begin{align*}
      V_1 &= \U_1 + \U_2 + \dots + \U_{k_1} = \lin \{\basis{1}{1}\} + \lin \{\basis{2}{1}\} + \dots + \lin \{\basis{k_1}{1}\} \\
          &= \lin \{\basis{1}{1}, \dots , \basis{k_1}{1}\} = \U = \V
    \end{align*}

    da $\flam{q_1-1} = \flam{0} = Id \in \LL(\V;\V) \implies \U = Id(\V) = \V$.
    Also muss $q_1 \ge 2$ gelten, wenn $V_1 \ne \V$.
  \end{tcolorbox}

  Weiter gilt für $l+1 \le i \le q_1, 1 \le j \le k_1$, dass die Vektoren $\flam{l}(\basis{j}{i})$ eine Basis
  von $\flam{l}(V_1)$ bilden, denn es gilt:

  \begin{align*}
    \flam{l}(\basis{j}{i}) &= \flam{l}(\flam{q_1-i}(b_j)) \\
                           &= \flam{q_1+l-i}(b_j) \\
                           &= \basis{j}{i-l}
  \end{align*}

  und

  \begin{align}
    \flam{l}(V_1) &= \lin \{\flam{l}(\basis{j}{i}) \mid 1 \le i \le q_1, 1 \le j \le k_1\} \\
                  &= \lin \{\flam{l}(\basis{j}{i}) \mid l+1 \le i \le q_1, 1 \le j \le k_1\}
  \end{align}

  d.h. die Vektoren sind linear unabhängig und ein Erzeugendensystem von $\flam{l}(V_1)$.

  
  Es gilt:
  \begin{align*}
    \flam{q_1-1}(V_1) &= \lin \{ \flam{q_1-1}(\basis{j}{i}) \mid 1 \le j \le k_1, 1 \le i \le q_1 \} \\
                      &\supseteq \lin \{ \flam{q_1-1}(\basis{j}{1}) \mid 1 \le j \le k_1\} = \U \\
                      &= \flam{q_1-1}(\V)
  \end{align*}
  und somit $V_1 \subset \V \implies \flam{q_1-1}(V_1) = \flam{q_1-1}(\V)$. Also existiert ein
  $q_2 \in \N$ mit $1 \le q_2 \le q_1-1$ und $\flam{q_2}(V_1) = \flam{q_2}(\V)$.
  Mit (1) und (2) folgt, dass $\flam{q_2-1}(\basis{j}{i})$ für
  $q_2 \le 1 \le q_1$, $1 \le i \le k_1$ eine Basis von $\flam{q_2-1}(V_1)$ ist.


  Betrachte nun die Abbildung:

  \begin{equation}
    \flam{} \colon \flam{q_2-1}(\V) \to \flam{q_2}(\V) = \flam{q_2}(V_1)
  \end{equation}

  Wegen $\V \neq V_1$ existeren $v \in \Kern \flam{}$ mit $v \in \V \backslash V_1$.
  Ergänze nun die Basis $\flam{q_2-1}(\basis{j}{i})$ von $\flam{q_2-1}(V_1)$ mit solchen
  Vektoren aus dem Kern der Abbildung zu einer Basis von $\flam{q_2-1}(\V)$.
  Seien dies die Vektoren $\flam{q_2-1}(b_{k_1+j}), 1 \le j \le k_2$.

  Diese Vektoren sind dann Eigenvektoren von $f$ und wie vorher definiere:

  \begin{equation*}
    \basis{k_1+j}{i} := \flam{q_2-i}(b_{k_1+j}), 0 \le i \le q_2, 1 \le j \le k2
  \end{equation*}

  \begin{tcolorbox}
  Dann gilt $\basis{k_1+j}{0} = \flam{q_2}(b_{k_1+j}) = 0$, da
  \begin{equation*}    
    \flam{q_2-1}(b_{k_1+j}) \in \Kern \flam{} \implies \flam{q_2}(b_{k_1+j}) = 0.
  \end{equation*}
  \end{tcolorbox}
  
  Sei $V_2 = \lin \{\basis{k_2+j}{i} \colon 1 \le i \le q2, 1 \le j \le k_2\}$. Dann folgt wie
  zuvor $\dim V_2 = k_2q_2$, und $\funcdown{f}{V_2}$ hat bzgl. der Basis $\basis{k_1+j}{i},
  1 \le i \le q_2, 1 \le j \le k_2$, eine Matrixdarstellung bestehend aus $k_2$ Jordanmatrizen
  der Form $J_{q_2}(\lam)$.

  Nun zeigen wir, dass die Vektoren
  \begin{equation}
    \basis{j}{i}, \betw{1}{i}{q_1}, \betw{1}{j}{k_1}, \space
    \basis{k_1+j}{i}, \betw{1}{i}{q_2}, \betw{1}{j}{k_2}
  \end{equation}
  linear unabängig sind und somit eine Basis von $V_1 \oplus V_2$. Dazu sei
  $0 = \sum_{i,j} \mu_{j}^{(i)}\basis{j}{i} + \sum_{i,j} \mu_{k_1+j}^{(i)}\basis{k_1+j}{i}$.
  Dann ist

  \begin{align*}
    0 &= \sum_{i,j} \mu_{j}^{(i)}\flam{q_2-1}(\basis{j}{i}) + \sum_{i,j} \mu_{k_1+j}^{(i)}\flam{q_2-1}(\basis{k_1+j}{i}) \\
      &= \sum_{i=1}^{q_1}\sum_{j=1}^{k_1} \mu_{j}^{(i)}\flam{q_2-1}(\basis{j}{i}) + \sum_{i,j} \mu_{k_1+j}^{(i)}\flam{q_2-1+q_2-i}(b_{k_1+j}) \\
      &= \sum_{i=q_2}^{q_1}\sum_{j=1}^{k_1} \mu_{j}^{(i)}\flam{q_2-1}(\basis{j}{i}) + \sum_{i,j} \mu_{k_1+j}^{(i)}\flam{q_2-1+q_2-i}(b_{k_1+j}) \\
      &= \sum_{i=q_2}^{q_1}\sum_{j=1}^{k_1} \mu_{j}^{(i)}\flam{q_2-1}(\basis{j}{i}) + \sum_{j=1}^{k_2} \mu_{k_1+j}^{(q_2)}\flam{q_2-1}(b_{k_1+j})
  \end{align*}

  Die in den Summer verbliebenen Vektoren bilden eine Basis von $\flam{q_2-1}(\V)$ und somit ist
  $\mu_{j}^{(i)}=0, \betw{q_2}{i}{q_1}, \betw{1}{j}{k_1}$ und $\mu_{k_1+j}^{(q_2)}=0, \betw{1}{j}{k_2}$. Somit ist also
  $0 = \sum_{i=1}^{q_2-1}\sum_{j=1}^{k_1}\mu_{j}^{(i)}\basis{j}{i}
  + \sum_{i=1}^{q_2-1}\sum_{j=1}^{k_2}\mu_{k_1+j}^{(i)}\basis{k_1+j}{i}$ und Anwendung von
  $\flam{q_2-2}$ auf diese Identität liefert als nächstes $\mu_{q_2-1} = 0, \betw{1}{j}{k_1},
  \mu_{k_1+j}^{q_2-1}=0, \betw{1}{j}{k_2}$, usw.

  Also hat $\funcdown{f}{V_1 \oplus V_2}$ bzgl. der Basis von (4) eine Matrixdarstellung
  bestehend aus $k_1$ Jordanmatrizen der Form $J_{q_1}(\lam)$ und $k_2$ Jordanmatrizen der Form
  $J_{q_2}(\lam)$.

  Gilt $\V = V_1 \oplus V_2$ so folgt die Aussage des Satzes sofort. Ansonsten sei $q_3 \le q_2-1$ minimal mit $\flam{q_3}(V_1 \oplus V_2) = \flam{q_3}(\V)$ und wie oben definiere
  einen Unterraum $V_3$ usw.
  
\end{proof}


\begin{example}
  Sei $A \in \M (3,3;\K)$ mit:
  \begin{equation*}
    A =
    \begin{bmatrix}
      1  & 1 & 0 \\
      -1 & 3 & 0 \\
      0  & 0 & 2 
    \end{bmatrix}
  \end{equation*}

  Dann gilt $P_A(t)=(t-2)^3$, und somit ist $\lam = 2$ ein Eigenwert von $A$. Es gilt, dass
  $q_1 = 2$ minimal mit $(A-\lam E)^{q_1}=0$ ist. Setze $\U := (A - \lam E)^{q_1-1}(\K^3)$ und
  $k_1 := \dim \U$. Daraus folgt $\U = (A - \lam E)^1(\K^3)$. Wir wissen:

  \begin{equation*}
    (A - \lam E) =
    \begin{bmatrix}
      -1 & 1 & 0 \\
      -1 & 1 & 0 \\
      0  & 0 & 0 
    \end{bmatrix}.
  \end{equation*}

  Also gilt:
  \begin{equation*}
    \U = \lin \{
    \begin{bmatrix}
      1 \\
      1 \\
      0  
    \end{bmatrix}
    \},
  \end{equation*}

  und $k_1 = 1$. Sei $b_1 = (1, 2, 0)^T$, dann ist $(A-2E)^{q_2-1}(1, 2, 0)^T = (1, 1, 0)^T$
  eine Basis von $\U$.
  Daraus folgt:

  \begin{align*}
    V_1 = \U_1 &= \lin \{\basis{1}{1}, \basis{1}{2}\} \\
               &= \lin \{ (A-2E)^{q_1-1}(1,2,0)^T, (A-2E)^{q_1-2}(1,2,0)^T\} \\ 
               &= \lin \{(1, 1, 0)^T, (1, 2, 0)^T\}.
  \end{align*}

  Also gilt $\dim V_1 = k_1q_1 = 1 \cdot 2 = 2$.

  Weiter ist $q_2 = 1$ minimal mit $(A-2E)^{q_2}(V_1) = (A-2E)^{q_2}(\V)$ und $q_2 \le q_1-1$,
  und es gilt $(A-2E)^{q_2-1}(V_1) = V_1$. Ergänzen wir nun die Basis von $V_1$ mit dem Vektor
  $v := (0, 0, 1)^T \in Kern (A-2E), v \in \V \backslash V_1$ dann bekommen wir eine Basis
  von $(A-2E)^{q_2-1}(\V)=\V$. Also ist $V_2 = \lin \{(0, 0, 1)^T\}$. Dann gilt
  $\V = V_1 \oplus V_2$, und somit hat $A$ die konjugierte Matrix C mit einer $J_{q_1}(2)$ und
  einer $J_{q_2}(2)$. Also:

  \begin{equation*}
    C =
    \begin{bmatrix}
      2 & 1 & 0 \\
      0 & 2 & 0 \\
      0 & 0 & 2 
    \end{bmatrix}
  \end{equation*}  
  
\end{example}
This proof was taken from the notes of Prof. Henk at the TU-Berlin. If you find any errors
or would like more explaination to some parts please create an issue at
\url{https://github.com/nullp0tr/lina2} or contact me through
\href{mailto:nullp0tr@gmail.com}{nullp0tr@gmail.com}. 
\end{document}
